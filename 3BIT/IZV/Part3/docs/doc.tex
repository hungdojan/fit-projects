\documentclass[a4paper,11pt]{article}
\usepackage[czech]{babel}
\usepackage[left=2cm,text={17cm,24cm},top=3cm]{geometry}
\usepackage[hidelinks]{hyperref}
\usepackage[utf8]{inputenc}
\usepackage[T1]{fontenc}
\usepackage{listings}
\usepackage{xcolor}
\usepackage[perpage]{footmisc}
\usepackage{graphicx}

\graphicspath{ {..} }

\definecolor{gray}{rgb}{0.95,0.95,0.95}
\definecolor{darkred}{rgb}{0.545,0,0}

\title{Dokumentace k projektu z IZV \\
        \large Hlavní důvody nehod na silnici}

\author{Hung Do \\ \href{mailto:xdohun00@stud.fit.vutbr.cz}{xdohun00@stud.fit.vutbr.cz}}

\begin{document}
    \maketitle
    Jaká je hlavní příčina nehod na silnicích v České republice? Podíváme-li se do databáze
    Policie ČR, můžeme vyčíst, že přes \textbf{56 \%} zaevidovaných nehod bylo způsobeno nesprávnou jízdou řidiče.

    Budeme-li zkoumat tento údaj podrobněji, můžeme zjistit, že nejčastější příčinou je \textbf{nevěnování se řízení vozidla}.
    Mezi dalšími důvody patří \textbf{nesprávné otáčení nebo couvání} nebo \textbf{nedodržení bezpečné vzdálenosti za vozidlem} (viz. \ref{table_accidents}).
    \begin{table}[ht]
        \centering
        \begin{tabular}{|c|c|c|} \hline
            \textbf{Příčina nehody}                             & \textbf{Počet nehod} & \textbf{Podíl (v \%)} \\ \hline
            Řidič se nevěnoval řízení vozidla                   & 96820 & 30.17 \\ \hline
            Nesprávné otáčení nebo couvání                      & 48920 & 15.24 \\ \hline
            Nedodržení bezpečné vzdálenosti za vozidlem         & 40631 & 12.66 \\ \hline
            Nezvládnnutí řízení vozidla                         & 31294 & 9.75 \\ \hline
            Vyhýbání bez dostatečného bočního odstupu (vůle)    & 23954 & 7.46 \\ \hline

        \end{tabular}
        \caption{Pět nejčastějších důvodů nehody}
        \label{table_accidents}
    \end{table}

    Přestože nevěnování se řízení je nejčastější důvod nehody, není to z daleka nejfatálnější. Nejčastěji totiž docházelo k úmrtí
    při \textbf{vjetí vozidla do protisměru}. V období mezi lety 2016 a (srpna) 2021
    přišlo tímto způsobem o život (do 24 hodin od nehody) 390 lidí (viz. \ref{figure_fatality}).

    Další zajímavou statistikou je průměrná škoda na vozidle dle příčiny nehody. Nejvíce poškozená vozidla jsou způsopbena
    \textbf{vjetím na nezpevněnou komunikaci} (cca 78 935.51 Kč).

    Nakonec se koukneme do krajů, jak si ony vedou v těchto statistikách. Například v Praze řidiči \textbf{nedodržují bezpečnou vzdálenost}, naopak \textbf{nejagresivněji}
    bývají řidiči v Moravskoslezkém kraji. A řidiči se nejvíce \textbf{nevěnují řízení} v Ústeckém kraji.
    \begin{figure}[ht]
        \centering
        \includegraphics[height=20cm]{fig}
        \caption{Počet zranění v závislosti na příčině nehody}
        \label{figure_fatality}
    \end{figure}


\end{document}
